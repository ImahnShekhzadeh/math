\section{Construction of $\mathbb R$}\label{app:completion_Q}

We will follow Ref. \cite{src:completion_of_Q} to construct the real numbers $\mathbb R$. There are two possibilities to do so:
\begin{enumerate}
	\item Dedekind completion: Every non-empty subset has a least upper bound (wrt $\leq$).
	\item Cauchy completion: Every Cauchy sequence converges wrt $\abs{\cdot}$.
\end{enumerate}

We will use the second definition of completion.

\begin{defn}
	The (archimedean) absolute value of $\mathbb Q$ is the function 
	\begin{align}
		\abs{\cdot}: \mathbb Q\to\mathbb{Q}_{\geq 0}, \abs{x} := \begin{cases}
			x &\text{if}\ x \geq 0,
			\\ -x &\text{if}\ x < 0.
		\end{cases}
	\end{align}
\end{defn}

\begin{defn}
	A \textit{Cauchy sequence} of rational numbers is a sequence $\seq[x_n]$ s.t. for every $\epsilon\in\mathbb{Q}_{> 0}$ there exists an $N = N(\epsilon)\in\mathbb N$ s.t. for all $m$, $n\geq N$, $\abs{x_m - x_n} < \epsilon$.
\end{defn}

\begin{theorem}\label{thrm:sum_Cauchy_sequences_Cauchy}
	Let $\seq[x_n]$ and $\seq[y_n]$ be two Cauchy sequences in $\mathbb Q$, then their sum $\seq[x_n + y_n]$ is also Cauchy.
\end{theorem}

\begin{proof}
	\Huge\color{red}TODO\color{black}\normalsize
\end{proof}

\begin{theorem}\label{thrm:prod_Cauchy_sequences_Cauchy}
	Let $\seq[x_n]$ and $\seq[y_n]$ be two Cauchy sequences in $\mathbb Q$, then their product $\seq[x_n \cdot y_n]$ is also Cauchy.
\end{theorem}

\begin{proof}
	\Huge\color{red}TODO\color{black}\normalsize
\end{proof}