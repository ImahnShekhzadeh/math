\section{Construction of $\mathbb R$}\label{app:completion_Q}

There are two possibilities to construct the real numbers $\mathbb R$:
\begin{enumerate}
	\item Dedekind completion: Every non-empty subset has a least upper bound (wrt $\leq$).
	\item Cauchy completion: Every Cauchy sequence converges wrt $\abs{\cdot}$.
\end{enumerate}

We will use the second definition of completion, and follow Ref. \cite{src:completion_of_Q,src:cauchy_construction_R}.

\begin{defn}
	The (archimedean) absolute value of $\mathbb Q$ is the function 
	\begin{align}
		\abs{\cdot}: \mathbb Q\to\mathbb{Q}_{\geq 0}, \abs{x} := \begin{cases}
			x &\text{if}\ x \geq 0,
			\\ -x &\text{if}\ x < 0.
		\end{cases}
	\end{align}
\end{defn}

\begin{defn}
	A \textit{Cauchy sequence} of rational numbers is a sequence $\seq[x_n]$ s.t. for every $\epsilon\in\mathbb{Q}_{> 0}$ there exists an $N = N(\epsilon)\in\mathbb N$ s.t. for all $m$, $n\geq N$, $\abs{x_m - x_n} < \epsilon$.
\end{defn}

\begin{theorem}\label{thrm:sum_Cauchy_sequences_Cauchy}
	Let $\seq[x_n]$ and $\seq[y_n]$ be two Cauchy sequences in $\mathbb Q$, then their sum $\seq[x_n + y_n]$ is also Cauchy.
\end{theorem}

\begin{proof}
	First, we note that the sum of two rational numbers is also rational. Since both $\seq[x_n]$ and $\seq[y_n]$ are Cauchy sequences, for all $\epsilon\in\mathbb Q_{>0}$ there is an $N'\in\mathbb N$ s.t. for all $m', n'\geq N$, we have $\abs{x_{m'} - x_{n'}} < \epsilon$ and an $\tilde{N}\in\mathbb N$ s.t. for all $\tilde{m}, \tilde{n}\geq \tilde{N}$, we have $\abs{y_{\tilde{m}} - y_{\tilde{n}}} < \epsilon$. Let $N := \max\{N', \tilde{N}\}$, then for all $m, n\geq N$, we have
	\begin{align*}
		\abs{x_m + y_m - (x_n + y_n)} = \abs{x_m - x_n + y_m - y_n} \leq \abs{x_m - x_n} + \abs{y_m - y_n} < 2\epsilon,
	\end{align*}
	which proves that $\seq[x_n + y_n]$ is also a Cauchy sequence.
\end{proof}

\begin{theorem}\label{thrm:prod_Cauchy_sequences_Cauchy}
	Let $\seq[x_n]$ and $\seq[y_n]$ be two Cauchy sequences in $\mathbb Q$, then their product $\seq[x_n \cdot y_n]$ is also Cauchy.
\end{theorem}

\begin{proof}
	First, we note that the multiplication and division of two rational numbers is rational. By Theorem \ref{thrm:cauchy_sequences_bounded}, Cauchy sequences are bounded, hence choose $M\in\mathbb Q$ s.t. $\abs{x_n}\leq M$ and $\abs{y_n}\leq M$ for all $n\in\mathbb N$, and note that for any $\epsilon\in\mathbb Q_{>0}$ there exists $N'\in\mathbb N$ s.t. for all $m', n'\geq N'$, we have $\abs{x_{m'} - x_{n'}} < \epsilon/M$ and an $\tilde{N}\in\mathbb N$ s.t. for all $\tilde{m}, \tilde{n}\geq \tilde{N}$, $\abs{y_{\tilde{m}} - y_{\tilde{n}}} < \epsilon/M$. Let $N := \max\{N', \tilde{N}\}$, then for all $m, n\geq N$, we have
	\begin{align*}
		\abs{x_my_m - x_ny_n} &= \abs{x_my_m + x_my_n - x_my_n - x_ny_n} = \abs{x_m\left(y_m - y_n\right) + y_n\left(x_m - x_n\right)}
		\\ &\leq \abs{x_m(y_m - y_n)} + \abs{y_n(x_m - x_n)} = \abs{x_m}\abs{y_m - y_n} + \abs{y_n}\abs{x_m - x_n}
		\\ &\leq M\cdot \epsilon/M + M\cdot \epsilon/M = 2\epsilon,
	\end{align*}
	where we used that $\abs{xy} = \abs{x}\abs{y}$. 
\end{proof}

\begin{remark}
	The set of Cauchy sequences in $\mathbb Q$ forms a ring, cf. Def. \ref{defn:ring}, where $0 = \seq[0]$ and $1 = \seq[1]$ are the neutral elements for the additive and multiplicative operation respectively. Since multiplication is commutative, we have a commutative ring.
\end{remark}

\begin{remark}
	The set of Cauchy sequences in $\mathbb Q$ does not form a field, cf. Def. \ref{defn:field}, though, since many Cauchy sequences do not have a multiplicative inverse, such as $(1, 0, 0, 0, \dots)$. Thus, the set of Cauchy sequences in $\mathbb Q$ does not form a field.
\end{remark}

\begin{defn}
	We say that a rational Cauchy sequence $\seq[x_n]$ is \textit{equivalent to zero} if $\lim\limits_{n\to\infty} \abs{x_n} = 0$. We say that two rational Cauchy sequences $\seq[x_n]$ and $\seq[y_n]$ are rational, denoted by $\seq[x_n] \sim \seq[y_n]$, if $\seq[x_n - y_n]$ is equivalent to $0$.
\end{defn}

\begin{theorem}
	The operation $\sim$ is an equivalence relation on the set of all Cauchy sequences in $\mathbb Q$.
\end{theorem}

\begin{proof}
	Note that any rational Cauchy sequence is equivalent to itself, since \\ $\seq[x_n - x_n] = \seq[0]$, and $\lim\limits_{n\to\infty}\abs{0} = 0$. We also have symmetry, since \\ $\abs{x_n - y_n} = \abs{y_n - x_n}$. Finally, if $\seq[x_n]\sim \seq[y_n]$ and $\seq[y_n]\sim \seq[z_n]$, then $\seq[x_n]\sim \seq[z_n]$, since $\abs{x_n - z_n} = \abs{x_n - y_n + y_n - z_n} \leq \abs{x_n - y_n} + \abs{y_n - z_n}$, thus the operation $\sim$ is also transitive.
	
	Thus, according to Def. \ref{defn:equivalence_relation}, $\sim$ forms an equivalence relation.
\end{proof}

\begin{defn}[Real numbers]
	We define $\mathbb R := \mathbb Q / \sim$. We extend the absolute value of $\mathbb Q$ to $\mathbb R$ via
	\begin{align}
		\abs{\left[\seq[x_n]\right]} := \left[ \seq[\abs{x_n}] \right],
	\end{align}
	where $\seq[x_n]$ is a Cauchy sequence in $\mathbb Q$.
\end{defn}

\begin{remark}
	We embed $\mathbb Q$ in $\mathbb R$ via the map $x\mapsto \left[\seq[x]\right]$, i.e. the equivalence class of a constant sequence, whose elements are always $x$. This map is injective. It is standard to abuse notation and write $x$ for $\left[\seq[x]\right]$. That way, we can see $\mathbb Q$ as a subset of $\mathbb R$.
\end{remark}

Before proving that $\mathbb Q$ is dense in $\mathbb R$ and that $\mathbb R$ is a field, we need to look at the algebraic structure of $\mathbb R$, and at what it means to say $\left[\seq[a_n]\right] < \left[\seq[b_n]\right]$.

\begin{defn}\label{defn:addition_multiplication_R}
	Let $s$, $t\in\mathbb R$, i.e. there are rational Cauchy sequences $\seq[s_n]$ and $\seq[t_n]$ s.t. $s = \left[\seq[s_n]\right]$ and $t = \left[\seq[t_n]\right]$.
	\begin{enumerate}[label=\alph*)]
		\item Define the map $+: \mathbb R\times \mathbb R\to\mathbb R, s + t := \left[\seq[s_n + t_n]\right]$, 
		\item and define $\cdot: \mathbb R\times \mathbb R\to\mathbb R, s \cdot t := \left[\seq[s_n\cdot t_n]\right]$.
	\end{enumerate}
\end{defn}

\begin{remark}
	Since we are dealing with equivalence classes, and $\seq[s_n]$ is merely a representation of $\left[\seq[s_n]\right]$, we need to verify that the above definitions are well-defined, i.e. representation-independent.
\end{remark}

\begin{theorem}
	The addition and multiplication of equivalence classes as defined in Def. \ref{defn:addition_multiplication_R} are well-defined.
\end{theorem}

\begin{proof}
	Let $s, t\in\mathbb R$, i.e. there exist rational Cauchy sequences $\seq[s_n]$ and $\seq[t_n]$ s.t. $s = \left[\seq[s_n]\right]$ and $t = \left[\seq[t_n]\right]$. Now let $\seq[s_n'], \seq[t_n']$ be rational Cauchy sequences s.t. $\seq[s_n]\sim \seq[s_n']$ and $\seq[t_n]\sim \seq[t_n']$, i.e. $\left[\seq[s_n]\right] = \left[\seq[s_n']\right]$ and  $\left[\seq[t_n]\right] = \left[\seq[t_n']\right]$. 
	
	Since 
	\begin{align*}
		\abs{s_n' + t_n' - (s_n + t_n)} \leq \abs{s_n' - s_n} + \abs{t_n' - t_n} \overset{n\to\infty}{\longrightarrow} 0,
	\end{align*}
	we have $\seq[s_n' + t_n] \sim \seq[s_n + t_n]$, and thus $\left[\seq[s_n' + t_n']\right] = \left[\seq[s_n + t_n]\right]$, which shows the well-definedness of $+$.
	
	For the multiplication $\cdot$, note that
	\begin{align*}
		\abs{s_n't_n' - s_nt_n} &= \abs{s_n't_n' + t_n's_n - t_n's_n - s_nt_n} = \abs{t_n'\left(s_n' - s_n\right) + s_n\left(t_n' - t_n\right)}
		\\ &\leq \abs{t_n'}\abs{s_n'-s_n} + \abs{s_n}\abs{t_n' - t_n}
		\\ &\leq M\abs{s_n'-s_n} + M\abs{t_n' - t_n} \overset{n\to\infty}{\longrightarrow} 0,
	\end{align*}
	where we proceeded as in the proof of Theorem \ref{thrm:prod_Cauchy_sequences_Cauchy} and chose an $M\in\mathbb Q$ s.t. $\abs{t_n'}\leq M$ and $\abs{s_n}\leq M$ for all $n\in\mathbb N$, cf. Theorem \ref{thrm:cauchy_sequences_bounded}. Thus, we have shown that $\seq[s_n't_n'] \sim \seq[s_nt_n]$, and hence $\left[\seq[s_n't_n']\right] = \left[\seq[s_nt_n]\right]$, showing the well-definedness of $\cdot$.
\end{proof}

\begin{theorem}
	$\mathbb R$, equipped with the two operations $+$ and $\cdot$, is a field.
\end{theorem}

\begin{proof}
	Here, we will only prove that every $s\in\mathbb R\backslash \{0\}$ has a multiplicative inverse (the rest is genereally easier to prove than this one).
	
	We shall first understand what the existence of the multiplicative inverse means. Let $\seq[s_n]$ be a rational Cauchy sequence s.t. $s = \left[\seq[s_n]\right]$. Since $s\ne 0$, we know that $\seq[s_n]\cancel{\sim}\seq[0]$, i.e. $\seq[s_n]$ does not converge to $0$. We now need to find a multiplicative inverse $t$ s.t. $s \cdot t = 1$. While this might seem trivial (since every $s_n\in\mathbb Q$ has a multiplicative inverse, i.e. we could choose $t_n = s_n^{-1}$), there is a subtle difficulty. The fact that $s$ is non-zero does \textit{not} mean that all sequence elements of $\seq[s_n]$ are unequal to zero. For example, let $s = 1$, then $\seq[s_n] = \left(0, 0, 0, 1, 1, 1, 1, 1, 1, 1, 1, \dots\right)$ is in the equivalence class of $[\seq[1]]$. However, the point is that \textit{eventually}, the sequence elements must be non-zero (since $s$ is non-zero), and thus there is an $N\in\mathbb N$ s.t. for $n\geq N$, we have $s_n\ne 0$. Thus, define a sequence $\seq[t_n]$ that for $n < N$ is zero and for $n\geq N$ let $t_n := s_n^{-1}$. Now the multiplicative inverse of $s$ is $t = \left[\seq[t_n]\right]$.
\end{proof}

\begin{defn}[Order of $\mathbb R$]\label{defn:order_R}
	Let $s\in\mathbb R$, i.e. there is a rational Cauchy sequence $\seq[s_n]$ s.t. $\left[\seq[s_n]\right] = s$. We say that $s$ is \textit{positive}, denoted by $s > 0$, if $s\ne 0$ and if there is an $N\in\mathbb N$ s.t. $s_n > 0$ for all $n\geq N$.
	
	Let $t\in\mathbb R$, i.e. there is a rational Cauchy sequence $\seq[t_n]$ s.t. $\left[\seq[t_n]\right] = t$. Then we write $s > t$ if $s - t$ is positive.
\end{defn}

\begin{theorem}
	The order of the real numbers is well-defined.
\end{theorem}

\begin{proof}
	Fix $s\in\mathbb R$, i.e. $s = \left[\seq[s_n]\right]$ for a rational Cauchy sequence $\seq[s_n]$. Also, let $s$ be positive, i.e. there is an $N\in\mathbb N$ s.t. $s_n > 0$ for all $n\geq N$. We need to show that for any other representative of $\left[\seq[s_n]\right]$, i.e. $\seq[s_n']\in\left[\seq[s_n]\right]$ with $\seq[s_n'] \sim \seq[s_n]$, there is an $N'\in\mathbb N$ s.t. $s_n' > 0$ for all $n\geq N'$.
	
	We first show that there exists a rational number $r\in\mathbb Q_{> 0}$ and an $\tilde{N}\in\mathbb N$ s.t. $s_n > r$ for all $n\geq \tilde{N}$ by contradiction. Suppose that for all $r\in\mathbb Q_{> 0}$ and all $\tilde{N}\in\mathbb N$ there exists an $n = n(r, \tilde{N})\geq \tilde{N}$ s.t. $s_n \leq r$. Then there is a subsequence $\left(s_{n_k}\right)_{k\in\mathbb N}$ of $\seq[s_n]$ s.t. $0 < s_{n_k} \leq 1/k$. We will prove this via induction. Set $n_1 := n(1, N)$. Assume we have constructed $n_1, n_2, \dots, n_k$ s.t. $n_1 < n_2 < \dots < n_k$ and $0 < s_{n_j} \leq 1/j$ for $j\in\{1, \dots, k\}$. Then set $n_{k + 1} := n(1/(k + 1), n_{k} + 1)$. We have $n_k < n_{k + 1}$ and $0 < s_{n_{k + 1}} \leq 1/(k +1)$. 
	
	Since $\seq[s_n]$ is Cauchy, there exists an $M_k\in\mathbb N$ s.t. $\abs{s_n - s_m} < 1/k$ for all $n, m\geq M_k$. Choose $l\geq k$ s.t. $n_l \geq M_k$. Then for $n\geq M_k$, we have
	\begin{align}
		\abs{s_n} = \abs{s_n - s_{n_l} + s_{n_l}} \leq \abs{s_n - s_{n_l}} + \abs{s_{n_l}} < \frac{1}{k} + \frac{1}{l} \leq \frac{2}{k} \overset{k\to\infty}{\longrightarrow} 0.
	\end{align}
	This means that $\seq[s_n]$ converges to $0$, and thus $s = 0$, which is a contradiction. Hence, there exists a rational number $r\in\mathbb Q_{> 0}$ and an $\tilde{N}\in\mathbb N$ s.t. $s_n > r$ for all $n\geq \tilde{N}$. Let $\seq[s_n']$ be another representative of $\left[\seq[s_n]\right]$, i.e. $\seq[s_n] \sim \seq[s_n']$, and thus $s_n' - s_n \overset{n\to\infty}{\longrightarrow} 0$. This in turn implies that $\abs{s_n' - s_n} < r$ for $n\geq N'$. Let $n\geq \max\{\tilde{N}, N'\}$, then we have
	\begin{align*}
		s_n' = s_n - (s_n - s_n') \geq s_n - \abs{s_n - s_n'} > r - r = 0,
	\end{align*}
	since $s_n - s_n' \leq \abs{s_n - s_n'}$, which implies $-(s_n - s_n') \geq - \abs{s_n - s_n'}$.
\end{proof}

\begin{theorem}
	Let $s, t, r\in\mathbb R$, then if $s > t$, we also have $s + r > t + r$.
\end{theorem}

\begin{proof}
	Let $\seq[s_n], \seq[t_n], \seq[r_n]$ be rational Cauchy sequences s.t. $s = \left[\seq[s_n]\right]$, $t = \left[\seq[t_n]\right]$ and $r = \left[\seq[r_n]\right]$. Since $s > t$, we know that there is an $N\in\mathbb N$ s.t. for all $n\geq N$, we have 
	\begin{align*}
		s_n - t_n > 0\Leftrightarrow s_n > t_n \Leftrightarrow s_n + r_n > t_n + r_n \Leftrightarrow \underbrace{(s_n + r_n) - (t_n + r_n)}_{=s_n - t_n} > 0
	\end{align*}
	Since $s_n - t_n$ does not go to $0$ for $n\to\infty$ (since $s > t$), we have that $(s_n + r_n) - (t_n + r_n)$ does not either, and thus $s + r > t + r$.
\end{proof}

The density of $\mathbb Q$ in $\mathbb R$ follows almost immediately from the construction of $\mathbb R$ from $\mathbb Q$.

\begin{theorem}\label{thrm:Q_dense_R}
	$\mathbb Q$ is dense in $\mathbb R$, i.e. for any $\epsilon \in\mathbb Q_{> 0}$ and $s\in\mathbb R$ there is a rational number $r\in\mathbb Q$ s.t. $\abs{s - r} < \epsilon$.
\end{theorem}

\begin{proof}
	Since $s\in\mathbb R$, there is a rational Cauchy sequence $\seq[s_n]$ s.t. $s = \left[\seq[s_n]\right]$. Since $\seq[s_n]$ is Cauchy, there is an $N\in\mathbb N$ s.t. for all $m, n\geq N$, we have $\abs{s_n - s_m} < \epsilon$. Let $r := s_N\in\mathbb R$, and note that we can embed it into $\mathbb R$ as $\left[\seq[a_N]\right]$. Thus, ${\abs{s - r} = \left[\seq[\abs{s_n - s_N}]\right]}$. For $n\geq N$, we have $\abs{s_n - s_N} < \epsilon$ and thus $s_n - s_N < \epsilon$ and $s_N - s_n < \epsilon$, since for any rational number $a$, $a \leq \abs{a}$. Thus, $\left(s_n - s_N\right) - \epsilon$ and $\left(s_N - s_n\right) - \epsilon$ are negative, i.e. $s - r < \epsilon$ and $r - s < \epsilon$, which can be summarized to $\abs{s - r¸} < \epsilon$.
\end{proof}

\begin{remark}
	The density of $\mathbb Q$ in $\mathbb R$ implies that we could replace $\epsilon\in\mathbb Q_{> 0}$ with $\epsilon\in\mathbb R_{>0}$ throughout.
\end{remark}

\begin{theorem}\label{thrm:R_Archimedean_property}
	$\mathbb R$ has the Archimedean property, i.e. for all $s, t\in\mathbb R$ with $s > 0$, there exists a natural number $m\in\mathbb N$ s.t. $m\cdot s > t$.
\end{theorem}

\begin{proof}
	Since $s, t\in\mathbb R$, there exist rational Cauchy sequences $\seq[s_n]$ and $\seq[t_n]$ s.t. $s = \left[\seq[s_n]\right]$ and $t = \left[\seq[t_n]\right]$. 
	
	Recally that by $m\in\mathbb N\subset \mathbb Q$, we mean the $m$ that is embedded into $\mathbb R$, i.e. $\left[\seq[m]\right]$. Thus, we need to show that there exists an $m, N\in\mathbb N$ s.t. $m s_n - t_n > 0$ for all $n\geq N$ and that $\seq[ms_n - t_n]$ does not converge to $0$, cf. Definition \ref{defn:order_R}.
	
	We will prove the first part by contradiction. Assume that for all $m, N\in\mathbb N$, there exists an $n\geq N$ s.t. $ms_n - t_n\leq 0$. For this, note that $\seq[t_n]$ is bounded, cf. \mbox{Theorem \ref{thrm:cauchy_sequences_bounded}}, i.e. there is an $M\in\mathbb Q$ s.t. $t_n \leq M$ for all $n\in\mathbb N$. By the Archimedean property of $\mathbb Q$, for any $\epsilon\in\mathbb Q_{> 0}$ there is an $m\in\mathbb N$ s.t. $m\epsilon/2 > M$, or equivalently $M/m < \epsilon/2$. Thus,
	\begin{align}\label{eq:archimedean_property_R}
		ms_n - t_n \leq 0 \Leftrightarrow s_n \leq \frac{t_n}{m} \leq \frac{M}{m} < \frac{\epsilon}{2}.
	\end{align}
	Since $\seq[s_n]$ is Cauchy, there exists an $N\in\mathbb N$ s.t. for all $n', k \geq N$, $\abs{s_k - s_{n'}} < \epsilon/2$, i.e. $s_k - s_{n'} < \epsilon/2$. Now choose an $n'\geq n\geq N$, then for all $k\geq N$, we have
	\begin{align*}
		s_k < s_{n'} + \frac{\epsilon}{2} \overset{\tiny\eqref{eq:archimedean_property_R}}{<} \epsilon.
	\end{align*}
	Thus, $s$ cannot be positive, contradicting the assumption that $s > 0$. Hence, we have shown that there exists an $m, N\in\mathbb N$ s.t. $m s_n - t_n > 0$ for all $n\geq N$.
	
	If $\seq[ms_n - t_n]$ converges to $0$, e.g. for $\seq[s_n] = \seq[1]$ and $\seq[t_n] = \seq[m]$, then we can choose $m + 1$ instead of $m$, since $(m + 1)s_n - t_n = ms_n - t_n + s_n \overset{n\to\infty}{\longrightarrow} \lim\limits_{n\to\infty}s_n > 0$, since $s \ne 0$.
\end{proof}

We can now prove that $\mathbb R$ is complete.

\begin{proposition}\label{prop:R_is_complete}
	Let $\seq[x_n]$ be a Cauchy sequence of \textit{real numbers}, then it converges to an $x\in\mathbb R$, i.e. $\mathbb R$ is complete.
\end{proposition}

\begin{proof}
	Fix an $x_n$ of the sequence. By the density of $\mathbb Q$ in $\mathbb R$, cf. Theorem \ref{thrm:Q_dense_R}, we know there exists a $q_n\in\mathbb Q$ s.t. $\abs{x_n - q_n} < 1/n$. We will now show that $\seq[q_n]$ is Cauchy in $\mathbb Q$. Let $\epsilon\in\mathbb Q_{> 0}$, then by the Archimedean property of $\mathbb Q$, there exists an $N\in\mathbb N$ s.t. $N\epsilon/3 > 1$, or equivalently $1/N < \epsilon/3$. Since $\seq[x_n]$ is Cauchy in $\mathbb R$, there exists an $M\in\mathbb R$ s.t. for all $n, m\geq M$, we have $\abs{x_n - x_m} < \epsilon/3$. Now, if $n, m\geq \max\{N, M\}$, we have 
	\begin{align*}
		\abs{q_n - q_m} &= \abs{q_n -x_n + x_n - x_m + x_m - q_m } \leq \abs{q_n - x_n} + \abs{x_n - x_m} + \abs{x_m - q_m} 
		\\  &< \frac{1}{n} + \frac{\epsilon}{3} + \frac{1}{m} \leq \frac{2}{N} + \frac{\epsilon}{3} < \epsilon,
	\end{align*}
	which proves that $\seq[q_n]$ is a Cauchy sequence, and thus it represents a real number, which we shall denote by $x$. We now want to show that $\seq[x_n]$ converges to $x$. First of all, by definition of $x$, $q_n - x\overset{n\to\infty}{\longrightarrow} 0$. We also know that $\abs{x_n - q_n} < 1/n$, cf. above. Thus, 
	\begin{align*}
		\abs{x_n - x} = \abs{x_n - q_n + q_n - x} \leq \abs{x_n - q_n} + \abs{q_n - x} < \frac{1}{n} + \abs{q_n - x}\overset{n\to\infty}{\longrightarrow} 0.
	\end{align*}
\end{proof}

\begin{remark}
	For the proof of the uniqueness of the completion of $\mathbb Q$, we refer to step 5 of the proof of Proposition \ref{prop:completion_of_metric_space_exists}.
\end{remark}

\begin{theorem}\label{thrm:between_two_reals_rational}
	Let $a, b\in\mathbb R$ s.t. $a < b$, then there is a rational number $q\in\mathbb Q$ s.t. $q\in (a, b)$.
\end{theorem}

\begin{proof}[Proof \cite{421600,3434503}]
	Since $\mathbb R$ has the Archimedean property, cf. Theorem \ref{thrm:R_Archimedean_property}, there exists an $N\in\mathbb N$ s.t. 
	\begin{align}\label{eq:between_two_reals_rational}
		N \cdot (b - a) > 1\Leftrightarrow Nb - Na > 1\Leftrightarrow Nb - 1 > Na.
	\end{align} 
	If $Nb\in\mathbb Z$, set $m := Nb - 1$, otherwise set $m := \floor{Nb} \in\mathbb Z$, and we have
	$$Nb > m \geq Nb - 1 \overset{\tiny\eqref{eq:between_two_reals_rational}}{>} Na\Rightarrow b > \frac{m}{N} > a,$$ where $m/N\in\mathbb Q$.
\end{proof}

\begin{corollary}
	Let $a, b\in\mathbb R$ s.t. $a < b$, then there are infinitely many rational numbers that are contained in $(a, b)$.
\end{corollary}

\begin{proof}
	Let $q_1\in\mathbb Q$ be s.t. $a < q_1 < b$, then we can construct a $q_2\in\mathbb Q$ s.t. $a < q_1 < q_2 < b$, now repeat this procedure infinitely many times.
\end{proof}