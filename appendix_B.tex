\section{Construction of $\mathbb R$}\label{app:completion_Q}

We will follow Ref. \cite{src:completion_of_Q} to construct the real numbers $\mathbb R$. There are two possibilities to do so:
\begin{enumerate}
	\item Dedekind completion: Every non-empty subset has a least upper bound (wrt $\leq$).
	\item Cauchy completion: Every Cauchy sequence converges wrt $\abs{\cdot}$.
\end{enumerate}

We will use the second definition of completion.

\begin{defn}
	The (archimedean) absolute value of $\mathbb Q$ is the function 
	\begin{align}
		\abs{\cdot}: \mathbb Q\to\mathbb{Q}_{\geq 0}, \abs{x} := \begin{cases}
			x &\text{if}\ x \geq 0,
			\\ -x &\text{if}\ x < 0.
		\end{cases}
	\end{align}
\end{defn}

\begin{defn}
	A \textit{Cauchy sequence} of rational numbers is a sequence $\seq[x_n]$ s.t. for every $\epsilon\in\mathbb{Q}_{> 0}$ there exists an $N = N(\epsilon)\in\mathbb N$ s.t. for all $m$, $n\geq N$, $\abs{x_m - x_n} < \epsilon$.
\end{defn}

\begin{theorem}\label{thrm:sum_Cauchy_sequences_Cauchy}
	Let $\seq[x_n]$ and $\seq[y_n]$ be two Cauchy sequences in $\mathbb Q$, then their sum $\seq[x_n + y_n]$ is also Cauchy.
\end{theorem}

\begin{proof}
	First, we note that the sum of two rational numbers is also rational. Since both $\seq[x_n]$ and $\seq[y_n]$ are Cauchy sequences, for all $\epsilon\in\mathbb Q_{>0}$ there is an $N'\in\mathbb N$ s.t. for all $m', n'\geq N$, we have $\abs{x_{m'} - x_{n'}} < \epsilon$ and an $\tilde{N}\in\mathbb N$ s.t. for all $\tilde{m}, \tilde{n}\geq \tilde{N}$, we have $\abs{y_{\tilde{m}} - y_{\tilde{n}}} < \epsilon$. Let $N := \max\{N', \tilde{N}\}$, then for all $m, n\geq N$, we have
	\begin{align*}
		\abs{x_m + y_m - (x_n + y_n)} = \abs{x_m - x_n + y_m - y_n} \leq \abs{x_m - x_n} + \abs{y_m - y_n} < 2\epsilon,
	\end{align*}
	which proves that $\seq[x_n + y_n]$ is also a Cauchy sequence.
\end{proof}

\begin{theorem}\label{thrm:prod_Cauchy_sequences_Cauchy}
	Let $\seq[x_n]$ and $\seq[y_n]$ be two Cauchy sequences in $\mathbb Q$, then their product $\seq[x_n \cdot y_n]$ is also Cauchy.
\end{theorem}

\begin{proof}
	First, we note that the multiplication and division of two rational numbers is rational. By Theorem \ref{thrm:cauchy_sequences_bounded}, Cauchy sequences are bounded, hence choose $M\in\mathbb Q$ s.t. $\abs{x_n}\leq M$ and $\abs{y_n}\leq M$ for all $n\in\mathbb N$, and note that for any $\epsilon\in\mathbb Q_{>0}$ there exists $N'\in\mathbb N$ s.t. for all $m', n'\geq N'$, we have $\abs{x_{m'} - x_{n'}} < \epsilon/M$ and an $\tilde{N}\in\mathbb N$ s.t. for all $\tilde{m}, \tilde{n}\geq \tilde{N}$, $\abs{y_{\tilde{m}} - y_{\tilde{n}}} < \epsilon/M$. Let $N := \max\{N', \tilde{N}\}$, then for all $m, n\geq N$, we have
	\begin{align*}
		\abs{x_my_m - x_ny_n} &= \abs{x_my_m + x_my_n - x_my_n - x_ny_n} = \abs{x_m\left(y_m - y_n\right) + y_n\left(x_m - x_n\right)}
		\\ &\leq \abs{x_m(y_m - y_n)} + \abs{y_n(x_m - x_n)} = \abs{x_m}\abs{y_m - y_n} + \abs{y_n}\abs{x_m - x_n}
		\\ &\leq M\cdot \epsilon/M + M\cdot \epsilon/M = 2\epsilon,
	\end{align*}
	where we used that $\abs{xy} = \abs{x}\abs{y}$. 
\end{proof}

\begin{remark}
	The set of Cauchy sequences in $\mathbb Q$ forms a ring, cf. Def. \ref{defn:ring}, where $0 = \seq[0]$ and $1 = \seq[1]$ are the neutral elements for the additive and multiplicative operation respectively. Since multiplication is commutative, we have a commutative ring.
\end{remark}

\begin{remark}
	The set of Cauchy sequences in $\mathbb Q$ does not form a field, cf. Def. \ref{defn:field}, though, since many Cauchy sequences do not have a multiplicative inverse, such as $(1, 0, 0, 0, \dots)$. Thus, the set of Cauchy sequences in $\mathbb Q$ does not form a field.
\end{remark}

\begin{defn}
	We say that a rational Cauchy sequence $\seq[x_n]$ is \textit{equivalent to zero} if $\lim\limits_{n\to\infty} \abs{x_n} = 0$. We say that two rational Cauchy sequences $\seq[x_n]$ and $\seq[y_n]$ are rational, denoted by $\seq[x_n] \sim \seq[y_n]$, if $\seq[x_n - y_n]$ is equivalent to $0$.
\end{defn}

\begin{theorem}
	The operation $\sim$ is an equivalence relation on the set of all Cauchy sequences in $\mathbb Q$.
\end{theorem}

\begin{proof}
	Note that any rational Cauchy sequence is equivalent to itself, since \\ $\seq[x_n - x_n] = \seq[0]$, and $\lim\limits_{n\to\infty}\abs{0} = 0$. We also have symmetry, since \\ $\abs{x_n - y_n} = \abs{y_n - x_n}$. Finally, if $\seq[x_n]\sim \seq[y_n]$ and $\seq[y_n]\sim \seq[z_n]$, then $\seq[x_n]\sim \seq[z_n]$, since $\abs{x_n - z_n} = \abs{x_n - y_n + y_n - z_n} \leq \abs{x_n - y_n} + \abs{y_n - z_n}$, thus the operation $\sim$ is also transitive.
	
	Thus, according to Def. \ref{defn:equivalence_relation}, $\sim$ forms an equivalence relation.
\end{proof}

\begin{defn}[Real numbers]
	We define $\mathbb R := \mathbb Q / \sim$. We extend the absolute value of $\mathbb Q$ to $\mathbb R$ via
	\begin{align}
		\abs{\left[\seq[x_n]\right]} := \left[ \seq[\abs{x_n}] \right],
	\end{align}
	where $\seq[x_n]$ is a Cauchy sequence in $\mathbb Q$.
\end{defn}

\begin{remark}
	We embed $\mathbb Q$ in $\mathbb R$ via the map $x\mapsto \left[\seq[x]\right]$, i.e. the equivalence class of a constant sequence, whose elements are always $x$. This map is injective. It is standard to abuse notation and write $x$ for $\left[\seq[x]\right]$. That way, we can see $\mathbb Q$ as a subset of $\mathbb R$.
\end{remark}

Before proving that $\mathbb Q$ is dense in $\mathbb R$ and that $\mathbb R$ is a field, we need to look at the algebraic structure of $\mathbb R$, and at what it means to say $\left[\seq[a_n]\right] < \left[\seq[b_n]\right]$.

\begin{defn}\label{defn:addition_multiplication_R}
	Let $s$, $t\in\mathbb R$, i.e. there are rational Cauchy sequences $\seq[s_n]$ and $\seq[t_n]$ s.t. $s = \left[\seq[s_n]\right]$ and $t = \left[\seq[t_n]\right]$.
	\begin{enumerate}[label=\alph*)]
		\item Define the map $+: \mathbb R\times \mathbb R\to\mathbb R, s + t := \left[\seq[s_n + t_n]\right]$, 
		\item and define $\cdot: \mathbb R\times \mathbb R\to\mathbb R, s \cdot t := \left[\seq[s_n\cdot t_n]\right]$.
	\end{enumerate}
\end{defn}

\begin{remark}
	Since we are dealing with equivalence classes, and $\seq[s_n]$ is merely a representation of $\left[\seq[s_n]\right]$, we need to verify that the above definitions are well-defined, i.e. representation-independent.
\end{remark}

\begin{theorem}
	The addition and multiplication of equivalence classes as defined in Def. \ref{defn:addition_multiplication_R} are well-defined.
\end{theorem}

\begin{proof}
	Let $s, t\in\mathbb R$, i.e. there exist rational Cauchy sequences $\seq[s_n]$ and $\seq[t_n]$ s.t. $s = \left[\seq[s_n]\right]$ and $t = \left[\seq[t_n]\right]$. Now let $\seq[s_n'], \seq[t_n']$ be rational Cauchy sequences s.t. $\seq[s_n]\sim \seq[s_n']$ and $\seq[t_n]\sim \seq[t_n']$, i.e. $\left[\seq[s_n]\right] = \left[\seq[s_n']\right]$ and  $\left[\seq[t_n]\right] = \left[\seq[t_n']\right]$. 
	
	Since 
	\begin{align*}
		\abs{s_n' + t_n' - (s_n + t_n)} = \abs{s_n' - s_n} + \abs{t_n' - t_n} \overset{n\to\infty}{\longrightarrow} 0,
	\end{align*}
	we have $\seq[s_n' + t_n] \sim \seq[s_n + t_n]$, and thus $\left[\seq[s_n' + t_n']\right] = \left[\seq[s_n + t_n]\right]$, which shows the well-definedness of $+$.
	
	For the multiplication $\cdot$, note that
	\begin{align*}
		\abs{s_n't_n' - s_nt_n} &= \abs{s_n't_n' + t_n's_n - t_n's_n - s_nt_n} = \abs{t_n'\left(s_n' - s_n\right) + s_n\left(t_n' - t_n\right)}
		\\ &\leq \abs{t_n'}\abs{s_n'-s_n} + \abs{s_n}\abs{t_n' - t_n}
		\\ &\leq M\abs{s_n'-s_n} + M\abs{t_n' - t_n} \overset{n\to\infty}{\longrightarrow} 0,
	\end{align*}
	where we proceeded as in the proof of Theorem \ref{thrm:prod_Cauchy_sequences_Cauchy} and chose an $M\in\mathbb Q$ s.t. $\abs{t_n'}\leq M$ and $\abs{s_n}\leq M$ for all $n\in\mathbb N$, cf. Theorem \ref{thrm:cauchy_sequences_bounded}. Thus, we have shown that $\seq[s_n't_n'] \sim \seq[s_nt_n]$, and hence $\left[\seq[s_n't_n']\right] = \left[\seq[s_nt_n]\right]$, showing the well-definedness of $\cdot$.
\end{proof}

\begin{theorem}
	$\mathbb Q$ is dense in $\mathbb R$.
\end{theorem}

\begin{proof}
	According to 
\end{proof}


\begin{remark}
	For the proof of the uniqueness of the completion of $\mathbb Q$, we refer to step 5 of the proof of Proposition \ref{prop:completion_of_metric_space_exists}.
\end{remark}