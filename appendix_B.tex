\section{Construction of $\mathbb R$}\label{app:completion_Q}

We will follow Ref. \cite{src:completion_of_Q} to construct the real numbers $\mathbb R$. There are two possibilities to do so:
\begin{enumerate}
	\item Dedekind completion: Every non-empty subset has a least upper bound (wrt $\leq$).
	\item Cauchy completion: Every Cauchy sequence converges wrt $\abs{\cdot}$.
\end{enumerate}

We will use the second definition of completion.

\begin{defn}
	The (archimedean) absolute value of $\mathbb Q$ is the function 
	\begin{align}
		\abs{\cdot}: \mathbb Q\to\mathbb{Q}_{\geq 0}, \abs{x} := \begin{cases}
			x &\text{if}\ x \geq 0,
			\\ -x &\text{if}\ x < 0.
		\end{cases}
	\end{align}
\end{defn}

\begin{defn}
	A \textit{Cauchy sequence} of rational numbers is a sequence $\seq[x_n]$ s.t. for every $\epsilon\in\mathbb{Q}_{> 0}$ there exists an $N = N(\epsilon)\in\mathbb N$ s.t. for all $m$, $n\geq N$, $\abs{x_m - x_n} < \epsilon$.
\end{defn}

\begin{theorem}\label{thrm:sum_Cauchy_sequences_Cauchy}
	Let $\seq[x_n]$ and $\seq[y_n]$ be two Cauchy sequences in $\mathbb Q$, then their sum $\seq[x_n + y_n]$ is also Cauchy.
\end{theorem}

\begin{proof}
	First, we note that the sum of two rational numbers is also rational. Since both $\seq[x_n]$ and $\seq[y_n]$ are Cauchy sequences, for all $\epsilon\in\mathbb Q_{>0}$ there is an $N'\in\mathbb N$ s.t. for all $m', n'\geq N$, we have $\abs{x_{m'} - x_{n'}} < \epsilon$ and an $\tilde{N}\in\mathbb N$ s.t. for all $\tilde{m}, \tilde{n}\geq \tilde{N}$, we have $\abs{y_{\tilde{m}} - y_{\tilde{n}}} < \epsilon$. Let $N := \max\{N', \tilde{N}\}$, then for all $m, n\geq N$, we have
	\begin{align*}
		\abs{x_m + y_m - (x_n + y_n)} = \abs{x_m - x_n + y_m - y_n} \leq \abs{x_m - x_n} + \abs{y_m - y_n} < 2\epsilon,
	\end{align*}
	which proves that $\seq[x_n + y_n]$ is also a Cauchy sequence.
\end{proof}

\begin{theorem}\label{thrm:prod_Cauchy_sequences_Cauchy}
	Let $\seq[x_n]$ and $\seq[y_n]$ be two Cauchy sequences in $\mathbb Q$, then their product $\seq[x_n \cdot y_n]$ is also Cauchy.
\end{theorem}

\begin{proof}
	First, we note that the multiplication and division of two rational numbers is rational. By Theorem \ref{thrm:cauchy_sequences_bounded}, Cauchy sequences are bounded, hence choose $M\in\mathbb Q$ s.t. $\abs{x_n}\leq M$ and $\abs{y_n}\leq M$ for all $n\in\mathbb N$, and note that for any $\epsilon\in\mathbb Q_{>0}$ there exists $N'\in\mathbb N$ s.t. for all $m', n'\geq N'$, we have $\abs{x_{m'} - x_{n'}} < \epsilon/M$ and an $\tilde{N}\in\mathbb N$ s.t. for all $\tilde{m}, \tilde{n}\geq \tilde{N}$, $\abs{y_{\tilde{m}} - y_{\tilde{n}}} < \epsilon/M$. Let $N := \max\{N', \tilde{N}\}$, then for all $m, n\geq N$, we have
	\begin{align*}
		\abs{x_my_m - x_ny_n} &= \abs{x_my_m + x_my_n - x_my_n - x_ny_n} = \abs{x_m\left(y_m - y_n\right) + y_n\left(x_m - x_n\right)}
		\\ &\leq \abs{x_m(y_m - y_n)} + \abs{y_n(x_m - x_n)} = \abs{x_m}\abs{y_m - y_n} + \abs{y_n}\abs{x_m - x_n}
		\\ &\leq M\cdot \epsilon/M + M\cdot \epsilon/M = 2\epsilon,
	\end{align*}
	where we used that $\abs{xy} = \abs{x}\abs{y}$. 
\end{proof}

\begin{remark}
	For the proof of the uniqueness of the completion of $\mathbb Q$, we refer to step 5 of the proof of Proposition \ref{prop:completion_of_metric_space_exists}.
\end{remark}